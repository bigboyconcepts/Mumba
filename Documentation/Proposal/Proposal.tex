\documentclass[letterpaper]{article}
    \usepackage{amsmath}
    \usepackage{tikz}
    \usepackage{epigraph}
    \usepackage{lipsum}
    \usepackage{glossaries}
    \usepackage{graphicx}
    \usepackage{imakeidx}
    \usepackage{hyperref}
    \graphicspath{{Images/}}
    \setcounter{tocdepth}{2}
    \makeglossaries
    \makeindex[columns=2, title=Alphabetical Index]
    \renewcommand\epigraphflush{flushright}
    \renewcommand\epigraphsize{\normalsize}
    \setlength\epigraphwidth{0.7\textwidth}

    \definecolor{titlepagecolor}{cmyk}{1,.60,0,.40}

    \DeclareFixedFont{\titlefont}{T1}{ppl}{b}{it}{0.5in}

    \makeatletter
    \def\printauthor{%
        {\large \@author}}
    \makeatother
    \author{
        Sam Miller \\
       }

    % The following code is borrowed from:  https://tex.stackexchange.com/a/86310/10898

    \newcommand\titlepagedecoration{%
    \begin{tikzpicture}[remember picture,overlay,shorten >= -10pt]

    \coordinate (aux1) at ([yshift=-15pt]current page.north east);
    \coordinate (aux2) at ([yshift=-410pt]current page.north east);
    \coordinate (aux3) at ([xshift=-4.5cm]current page.north east);
    \coordinate (aux4) at ([yshift=-150pt]current page.north east);

    \begin{scope}[titlepagecolor!40,line width=12pt,rounded corners=12pt]
    \draw
      (aux1) -- coordinate (a)
      ++(225:5) --
      ++(-45:5.1) coordinate (b);
    \draw[shorten <= -10pt]
      (aux3) --
      (a) --
      (aux1);
    \draw[opacity=0.6,titlepagecolor,shorten <= -10pt]
      (b) --
      ++(225:2.2) --
      ++(-45:2.2);
    \end{scope}
    \draw[titlepagecolor,line width=8pt,rounded corners=8pt,shorten <= -10pt]
      (aux4) --
      ++(225:0.8) --
      ++(-45:0.8);
    \begin{scope}[titlepagecolor!70,line width=6pt,rounded corners=8pt]
    \draw[shorten <= -10pt]
      (aux2) --
      ++(225:3) coordinate[pos=0.45] (c) --
      ++(-45:3.1);
    \draw
      (aux2) --
      (c) --
      ++(135:2.5) --
      ++(45:2.5) --
      ++(-45:2.5) coordinate[pos=0.3] (d);
    \draw
      (d) -- +(45:1);
    \end{scope}
    \end{tikzpicture}%
    }




\begin{document}
\begin{titlepage}

	\noindent
	\titlefont CS4610 \\Project Management Plan\par
	\epigraph{Project Proposal for Mumba\\ October 01, 2018}%
	{\textit{}\\ \textsc{}}
	\null\vfill
	\vspace*{1cm}
	\noindent
	\hfill
	\begin{minipage}{0.35\linewidth}
		\begin{flushright}
			\printauthor
		\end{flushright}
	\end{minipage}
	%
	\begin{minipage}{0.02\linewidth}
		\rule{1pt}{125pt}
	\end{minipage}
	\titlepagedecoration
\end{titlepage}
\tableofcontents
\pagebreak

\section{Introduction}

\subsection{Project Overview}

Mumba will be a kanban board like task manager tailored toward smaller projects and assignments. It will be hosted at https://mumba.goldenticket.biz  (I already own the domain name *.goldenticket.biz and dont want to buy another at this time). I hope for this project to be extensible to leverage apis from github/gitlab and google calendar if time permits (premium features on similar products such as Trello)


\subsection{Project Deliverables}
https://mumba.goldenticket.biz should deliver:
\begin{itemize}
  \item Entity-Relationship model database and diagram for Users Boards Lists and cards
  \item User registration
    \begin{itemize}
      \item At the very least registration will be required and consist of email password combinations.
      \item If possible I will implement the option to sign in with other authentication providers such as google.
    \end{itemize}
  \item a login page
  \item a dashboard page (a list of boards)
  \item a board view page (a listing of lists and cards under lists)
  \item the ability to create and modify:
    \begin{itemize}
      \item Boards
      \item Lists
      \item Cards
    \end{itemize}
\end{itemize}

\pagebreak
\section{Project Organization}


\subsection{Tools and Techniques}
To implement this project I am planning to leverage the asp dotnet core framework. In order to manage the pages and content I will be using the Model View Controller pattern to manage my webpages. I will use a kesteral web-server behind an apache server to host the static content, a dotnet runtime application to handle the dynamic content and an SQL based server (possibly mariaDB) to hold my database elements. I will be implementing some form of authentication more than likely openID. I will use letsencrypt to get a ssl certificate. I plan on hosting the apache/kesteral combination in Azure for its support of dotnet. I may also end up using one of Azures Mysql implementations instead of mariaDB. If time permits I may implement continuous deployment and other fun dev-ops projects.

\pagebreak
\section{Project Management Plan}

\subsection{Tasks}
\begin{itemize}
  \item Create E-R diagram
  \begin{itemize}
    \item Description: Create a model described the database schema for for Users Boards Lists and Cards
    \item Deliverables: The E-R diagram.
  \end{itemize}
  \item Create static web content templates
  \begin{itemize}
    \item Description: A template for all webpages that will be included in the project, CSS theme etc.
    \item Deliverables: HTML and CSS templates for the various pages minus the dynamic content.
  \end{itemize}
  \item implement User registration and login
  \begin{itemize}
    \item Description: Implement the database and api required for a login page, create the login page and test.
    \item Deliverables: Dynamic login page and secure authentication for mumba.goldenticket.biz
  \end{itemize}
  \item implement database link
  \begin{itemize}
    \item Description: Implement the EM migrations needed for the asp.net app to integrate with the database.
    \item Deliverables: Functional relational database entries back-end view of said database
  \end{itemize}
  \item create first board and board creation method
  \begin{itemize}
    \item Description: Implement the dashboard for boards, create the Board view and board creation view and controllers
    \item Deliverables: Web page that shows a users existing boards with the ability to create new boards
  \end{itemize}
  \item Create first list and list creation method
  \begin{itemize}
    \item Description: In the view of a board generate the dynamic content that is lists
    \item Deliverables: Boards can now have 0 - many lists
  \end{itemize}
  \item Create first card and card creation method
  \begin{itemize}
    \item Description: Create the cards and the detailed pop up of the cards and display them in the associated list
    \item Deliverables: Lists can now have 0 - many cards and cards will have fillable fields:
      \begin{itemize}
        \item title
        \item description
        \item due date
        \item stretch goals:
        \begin{itemize}
          \item Calendar view of cards / google calendar integration
          \item to-do lists on cards
          \item Git integration to tag branches, releases, commits, or issues to cards
        \end{itemize}
      \end{itemize}
  \end{itemize}
  \item Write project Report and prepare documentation
    \begin{itemize}
      \item Description: Project report as described in the project description appendix and a demonstration of the delivered product in class.
      \item  Deliverables: Project Report and Project Presentation
    \end{itemize}
\end{itemize}

\subsection{Timetable}
\begin{itemize}
  \item Week 1: Start on E-R Diagram and finish it and any other necessary documentation.
  \item Week 2: Create Basic project layout and begin static content.
  \item Week 3: Implement basic login system
  \item Week 4: Create Board dashboard
  \item Week 5: Begin creation of functional board
  \item Week 6: Continue board creation and start list creation
  \item Week 7: Add functionality to cards
  \item Week 8: Polish and write up reports
\end{itemize}
\end{document}
